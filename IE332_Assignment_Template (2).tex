\documentclass[11pt]{article}

%  USE PACKAGES  ---------------------- 
\usepackage[margin=0.7in,vmargin=1in]{geometry}
\usepackage{amsmath,amsthm,amsfonts}
\usepackage{amssymb}
\usepackage{fancyhdr}
\usepackage{enumerate}
\usepackage{mathtools}
\usepackage{hyperref,color}
\usepackage{enumitem,amssymb}
\newlist{todolist}{itemize}{4}
\setlist[todolist]{label=$\square$}
\usepackage{pifont}
\newcommand{\cmark}{\ding{51}}%
\newcommand{\xmark}{\ding{55}}%
\newcommand{\done}{\rlap{$\square$}{\raisebox{2pt}{\large\hspace{1pt}\cmark}}%
\hspace{-2.5pt}}
\newcommand{\HREF}[2]{\href{#1}{#2}}
\usepackage{textcomp}
\usepackage{listings}
\lstset{
basicstyle=\small\ttfamily,
% columns=flexible,
upquote=true,
breaklines=true,
showstringspaces=false
}
%  -------------------------------------------- 

%  HEADER AND FOOTER (DO NOT EDIT) ----------------------
\newcommand{\problemnumber}{0}
\pagestyle{fancy}
\fancyhead{}
\fancyhead[L]{\textbf{ \problemnumber}}
\newcommand{\newquestion}[1]{
\clearpage % page break and flush floats
\renewcommand{\problemnumber}{#1} % set problem number for header
\phantom{}  % Put something on the page so it shows
}
\fancyfoot[L]{IE 332}
\fancyfoot[C]{Assignment submission}
\fancyfoot[R]{Page \thepage}
\renewcommand{\footrulewidth}{0.4pt}

%  --------------------------------------------


%  COVER SHEET (FILL IN THE TABLE AS INSTRUCTED IN THE ASSIGNMENT) ----------------------
\newcommand{\addcoversheet}{
\clearpage
\thispagestyle{empty}
\vspace*{0.5in}

\begin{center}
\Huge{{\bf IE332 Project \#2}} % <-- replace with correct assignment #

Due: Month Date 28th, 11:59pm EST % <-- replace with correct due date and time
\end{center}

\vspace{0.3in}

\noindent We have {\bf read and understood the assignment instructions}. We certify that the submitted work does not violate any academic misconduct rules, and that it is solely our own work. By listing our names below we acknowledge that any misconduct will result in appropriate consequences. 

\vspace{0.2in}

\noindent {\em ``As a Boilermaker pursuing academic excellence, I pledge to be honest and true in all that I do.
Accountable together -- we are Purdue.''}

\vspace{0.3in}

\begin{table}[h!]
  \begin{center}
    \label{tab:table1}
    \begin{tabular}{c|ccccc|c|c}
      Student & Development & Analysis & Implementation & Testing & Report & Overall & DIFF\\
      \hline
      Ashwin Jayaraj & 20 & 20 & 20 & 20 & 20 & 100 & 0\\ 
      Delaney McGinty & 20 & 20 & 20 & 20 & 20 & 100 & 0\\
      Sage Nichols & 20 & 20 & 20 & 20 & 20 & 100 & 0\\
      Jacob Schneider & 20 & 20 & 20 & 20 & 20 & 100 & 0\\
      Aditya Vijendran & 20 & 20 & 20 & 20 & 20 & 100 & 0\\
      \hline
      St Dev & 0 & 0 & 0 & 0 & 0 & 0 & 0
    \end{tabular}
  \end{center}
\end{table}

\vspace{0.2in}

\noindent Date: \today.
}
%  -----------------------------------------

%  TODO LIST (COMPLETE THE FULL CHECKLIST - USE AS EXAMPLE THE FIRST CHECKED BOXES!) ----------------------
\newcommand{\addtodo}{
\clearpage
\thispagestyle{empty}

\section*{Read Carefully. Important!}

\noindent By electronically uploading this assignment to Brightspace you acknowledge these statements and accept any repercussions if in any violation of ANY Purdue Academic Misconduct policies. You must upload your homework on time for it to be graded. No late assignments will be accepted. {\bf Only the last uploaded version of your assignment before the due date will be graded}.

\vspace{0.2in}

\noindent {\bf NOTE:} You should aim to submit no later than 30 minutes before the deadline, as there could be last minute network traffic that would cause your assignment to be late, resulting in a grade of zero. 

\vspace{0.2in}

\noindent When submitting your assignment it is assumed that every student considers the below checklist, as there are grading consequences otherwise (e.g., not submitting a cover sheet is an automatic grade of ZERO).

\begin{todolist}

    \item[\done] Your solutions were prepared using the \LaTeX template provided in Brightspace. 
    \item[\done] Your submission has a cover sheet as its first page and this checklist as its second page, according to the template provided.
	 \item All of your solutions (program code, etc.) are included in the submission as requested. % Check this checkbox and the following ones if satisfied <---
    \item[\done] You have not included any screen shots, photos, etc. (plots should be intermediately saved as .png files and then added into your .tex file). % <---
	 \item[\done] All math notation and algorithms (algorithmic environment) are created using appropriate \LaTeX code (no pictures, handwritten solutions, etc.). % <---
    \item[\done] The .pdf is submitted as an individual file and not in a {\tt .zip}.
    \item[\done] You kept the \LaTeX source code in your files until this assignment is graded, in case you are required to show proof of creating your assignment using \LaTeX.  % <---
    \item[\done] If submitting with a partner, your partner is added in the submission section in Gradescope after you upload your file. % <---
    \item[\done] You have correctly matched each question to its page \# in the .pdf submission in the Gradescope section (after you uploaded your file).
    \item[\done] Watch videos on creating pseudocode if you need a refresher or quick reference to the idea. These are good starter videos:    % <---
    
     \HREF{https://www.youtube.com/watch?v=4jLO0vXPktU}{www.youtube.com/watch?v=4jLO0vXPktU} 
    
    \HREF{https://www.youtube.com/watch?v=yGvfltxHKUU}{www.youtube.com/watch?v=yGvfltxHKUU}
\end{todolist}
}

%% LaTeX
% Für alle, die die Schönheit von Wissenschaft anderen zeigen wollen
% For anyone who wants to show the beauty of science to others

%  -----------------------------------------


\begin{document}


\addcoversheet
\addtodo

% BEGIN YOUR ASSIGNMENT HERE:

% Question 1
\newquestion{Main}

\paragraph{Introduction} \hspace{0pt} \\

With the rise of machine learning, image classifiers have become an integral part of automatization in many industries: medicine, security, and transportation, for example. Despite this, the classifiers are flawed; many algorithms can exploit these flaws and cause the misclassification of objects. Our team has researched many of these algorithms and designed an adversarial attack to fool the given classifier. Our attack consists of five sub-algorithms coordinated by a main algorithm to improve the likeliehood of fooling the classifier with any given image. By attempting to break the model, we have gained a better understanding of how image classification and machine learning algorithms work under the surface.

\section{Sub-Algorithms}

We developed five sub-algorithms for our main algorithm to choose from. Each one uses a different method to try and fool the classifier, so the diversity of pros and cons gives our main algorithm the flexibility to choose the most effective for a given image.

\paragraph{Fast Gradient Sign Method} \hspace{0pt} \\

The Fast Gradient Sign Method (FGSM) is a white box adversarial attack that utilizes the gradient of the cost function, taken with respect to the input. Because the cost function of the model is needed, this form of attack can only be used where access to the model is given. This method, even with changing less than one percent of pixels in the image, can fool a classifier with over 99\% confidence (Goodfellow et al., 2014). Due to the information, and the relative simplicity and universality of the algorithm, the FGSM was a sound choice for one of our sub-algorithms. 

\paragraph{Projected Gradient Descent} \hspace{0pt} \\

The Projected Gradient Descent (PGD) method is another type of white box iterative attack, meaning the attacker can access a copy of the victim model’s weights. In a PGD attack, the attacker calculates the gradient of the loss function iteratively with respect to the input. It then disturbs the input in the direction of this known gradient. To keep the resulting adversarial example acceptable to the model, the disturbance is projected onto the set of allowable inputs. Because this attack is iterative, it is able to fine-tune weights to ensure it is maximizing the disturbances' impact while also minimizing its detectability. This two-front attack makes PGD attacks one of the most effective methods to fool a machine learning model.

\section{Weight Optimization}

\paragraph{Hill Climbing Optimization} \hspace{0pt} \\

Hill climbing is an optimization method that iteratively evaluates the neighboring solutions of the current solution and selects the best one. If the algorithm finds a new solution, it will move to that and do that for the new solution. This process is then repeated until a solution is found where no further improvement can be made. As far as adversarial attacks go, hill climbing can be used to optimize the perturbations made to an image based on the pixel budget. The perturbation will move through the image and iteratively update it to fool the classifier even more with each iteration. This will further confuse the algorithm until there is no better solution. The downside to using this is that if the algorithm reaches a local optimal solution, it can get stuck there. Since it is being compared to nearby solutions and not all solutions, it can get stuck in this position as it does not identify any nearby solutions.

\paragraph{Random Search Optimization} \hspace{0pt} \\

Random search optimization is a very straightforward optimization method that selects random potential solutions and evaluates their performance. It does not rely on nearby local solutions; it relies on the global solutions of the space. The algorithm will keep generating random solutions until a certain constraint blocks it from continuing. In the context of adversarial attacks, the random search can produce random perturbations of an image and then evaluate the performance of that perturbation. This can then be optimized to find the best-performing perturbation, and it will continue to find perturbations until a better one is found. The only downside to this is that if the image is large, it will require a large number of iterations. Even if the iterations are high, it will take a lot of computational power to run through that many iterations as the algorithm does not run locally, it runs globally throughout the whole image.

\paragraph{Genetic Optimization} \hspace{0pt} \\

Genetic optimization is an optimization algorithm that references natural evolution to find an optimal solution. It does this by having a population of potential solutions and applying evolutionary operations such as mutations and crossovers. In the context of adversarial attacks, genetic optimization can be used to optimize the perturbations made to an image to maximize the success rate of the attacks. It first finds a population of random perturbations. It then applies the perturbations to an image and evaluates the performance. From these evaluations, it will find the best-performing ones and perform either a mutation or a crossover. The mutation will introduce a new random change in the perturbation. The crossover will combine any 2 random perturbations. Using either one of these, it will update this back into the original image and continue these steps for however many iterations, or until an acceptable solution is found. The downfall to this is that you must manually input and tune the parameters, such as the population size, probabilities for crossovers or mutations, and evaluation of performance. It can also be computationally powerful, making it less efficient.

\newquestion{Main}

\begin{center}
\textbf{References}
\end{center}

\begin{hangparas}

Goodfellow, I. J., Shlens, J., \& Szegedy, C. (2014). \textit{Explaining and harnessing adversarial examples. arXiv preprint arXiv:1412.6572.}

\end{hangparas}

\newquestion{Appendix A}
\begin{center}
\textbf{Appendix A: Solution}
\end{center}

\newquestion{Appendix B}
\begin{center}
\textbf{Appendix B: Testing, Verification, and Correctness}
\end{center}

\newquestion{Appendix C}
\begin{center}
\textbf{Appendix C: Runtime Complexity and Walltime}
\end{center}

\newquestion{Appendix D}
\begin{center}
\textbf{Appendix D: Performance}
\end{center}


\end{document}

